\documentclass[compress]{beamer}

% The following defines a beamer theme. Uncomment or delete to avoid option 
% clashes with other beamer themes.
\usepackage{.fva_templates/freiburg}

% Overwrite the beamer footline with the FVA corporate graphics.
% Make sure this comes after the beamer theme selections. 
\usepackage{.fva_templates/fva}
\setbeamertemplate{caption}[numbered]
\setbeamertemplate{caption label separator}{: }
\setbeamercolor{caption name}{fg=normal text.fg}
\beamertemplatenavigationsymbolsempty
\usepackage{lmodern}
\usepackage{amssymb,amsmath}
\usepackage{ifxetex,ifluatex}
\usepackage{fixltx2e} % provides \textsubscript
\ifnum 0\ifxetex 1\fi\ifluatex 1\fi=0 % if pdftex
  \usepackage[T1]{fontenc}
  \usepackage[utf8]{inputenc}
\else % if luatex or xelatex
  \ifxetex
    \usepackage{mathspec}
  \else
    \usepackage{fontspec}
  \fi
  \defaultfontfeatures{Ligatures=TeX,Scale=MatchLowercase}
\fi
% use upquote if available, for straight quotes in verbatim environments
\IfFileExists{upquote.sty}{\usepackage{upquote}}{}
% use microtype if available
\IfFileExists{microtype.sty}{%
\usepackage{microtype}
\UseMicrotypeSet[protrusion]{basicmath} % disable protrusion for tt fonts
}{}
\newif\ifbibliography
\usepackage{color}
\usepackage{fancyvrb}
\newcommand{\VerbBar}{|}
\newcommand{\VERB}{\Verb[commandchars=\\\{\}]}
\DefineVerbatimEnvironment{Highlighting}{Verbatim}{commandchars=\\\{\}}
% Add ',fontsize=\small' for more characters per line
\newenvironment{Shaded}{}{}
\newcommand{\KeywordTok}[1]{\textcolor[rgb]{0.00,0.44,0.13}{\textbf{{#1}}}}
\newcommand{\DataTypeTok}[1]{\textcolor[rgb]{0.56,0.13,0.00}{{#1}}}
\newcommand{\DecValTok}[1]{\textcolor[rgb]{0.25,0.63,0.44}{{#1}}}
\newcommand{\BaseNTok}[1]{\textcolor[rgb]{0.25,0.63,0.44}{{#1}}}
\newcommand{\FloatTok}[1]{\textcolor[rgb]{0.25,0.63,0.44}{{#1}}}
\newcommand{\ConstantTok}[1]{\textcolor[rgb]{0.53,0.00,0.00}{{#1}}}
\newcommand{\CharTok}[1]{\textcolor[rgb]{0.25,0.44,0.63}{{#1}}}
\newcommand{\SpecialCharTok}[1]{\textcolor[rgb]{0.25,0.44,0.63}{{#1}}}
\newcommand{\StringTok}[1]{\textcolor[rgb]{0.25,0.44,0.63}{{#1}}}
\newcommand{\VerbatimStringTok}[1]{\textcolor[rgb]{0.25,0.44,0.63}{{#1}}}
\newcommand{\SpecialStringTok}[1]{\textcolor[rgb]{0.73,0.40,0.53}{{#1}}}
\newcommand{\ImportTok}[1]{{#1}}
\newcommand{\CommentTok}[1]{\textcolor[rgb]{0.38,0.63,0.69}{\textit{{#1}}}}
\newcommand{\DocumentationTok}[1]{\textcolor[rgb]{0.73,0.13,0.13}{\textit{{#1}}}}
\newcommand{\AnnotationTok}[1]{\textcolor[rgb]{0.38,0.63,0.69}{\textbf{\textit{{#1}}}}}
\newcommand{\CommentVarTok}[1]{\textcolor[rgb]{0.38,0.63,0.69}{\textbf{\textit{{#1}}}}}
\newcommand{\OtherTok}[1]{\textcolor[rgb]{0.00,0.44,0.13}{{#1}}}
\newcommand{\FunctionTok}[1]{\textcolor[rgb]{0.02,0.16,0.49}{{#1}}}
\newcommand{\VariableTok}[1]{\textcolor[rgb]{0.10,0.09,0.49}{{#1}}}
\newcommand{\ControlFlowTok}[1]{\textcolor[rgb]{0.00,0.44,0.13}{\textbf{{#1}}}}
\newcommand{\OperatorTok}[1]{\textcolor[rgb]{0.40,0.40,0.40}{{#1}}}
\newcommand{\BuiltInTok}[1]{{#1}}
\newcommand{\ExtensionTok}[1]{{#1}}
\newcommand{\PreprocessorTok}[1]{\textcolor[rgb]{0.74,0.48,0.00}{{#1}}}
\newcommand{\AttributeTok}[1]{\textcolor[rgb]{0.49,0.56,0.16}{{#1}}}
\newcommand{\RegionMarkerTok}[1]{{#1}}
\newcommand{\InformationTok}[1]{\textcolor[rgb]{0.38,0.63,0.69}{\textbf{\textit{{#1}}}}}
\newcommand{\WarningTok}[1]{\textcolor[rgb]{0.38,0.63,0.69}{\textbf{\textit{{#1}}}}}
\newcommand{\AlertTok}[1]{\textcolor[rgb]{1.00,0.00,0.00}{\textbf{{#1}}}}
\newcommand{\ErrorTok}[1]{\textcolor[rgb]{1.00,0.00,0.00}{\textbf{{#1}}}}
\newcommand{\NormalTok}[1]{{#1}}

% Prevent slide breaks in the middle of a paragraph:
\widowpenalties 1 10000
\raggedbottom

\AtBeginPart{
  \let\insertpartnumber\relax
  \let\partname\relax
  \frame{\partpage}
}
\AtBeginSection{
  \ifbibliography
  \else
    \let\insertsectionnumber\relax
    \let\sectionname\relax
    \frame{\sectionpage}
  \fi
}
\AtBeginSubsection{
  \let\insertsubsectionnumber\relax
  \let\subsectionname\relax
  \frame{\subsectionpage}
}

\setlength{\emergencystretch}{3em}  % prevent overfull lines
\providecommand{\tightlist}{%
  \setlength{\itemsep}{0pt}\setlength{\parskip}{0pt}}
\setcounter{secnumdepth}{0}

\title{Funktionen in \textbf{R}}
\subtitle{Ein Einstieg an der Forstlichen Versuchs- und Forschungsanstalt
Baden-Württemberg.}
\author{Dominik Cullmann}
\date{}

\begin{document}
\frame{\titlepage}

\section{Funktionen in R}\label{funktionen-in-r}

\begin{frame}[fragile]{Was sind Funktionen?}

\begin{itemize}
\tightlist
\item
  Funktionen sind Programmkonstrukte, mit denen Du Teile des von Dir
  geschriebenen Codes wiederverwenden kannst (Siehe
  \href{https://de.wikipedia.org/wiki/Funktion_(Programmierung)}{Wikipedia}).
\item
  R-Funktionen kennst Du wahrscheinlich schon: \texttt{sum()},
  \texttt{mean()}, \texttt{summary()}
\end{itemize}

\end{frame}

\begin{frame}{Warum Funktionen?}

\begin{itemize}
\tightlist
\item
  Damit Du Teile des von Dir geschriebenen Codes wiederverwenden kannst.
\item
  Dann musst Du, wenn Du Fehler im Code entdeckst, diesen auch nur an
  einer Stelle korrigieren.
\end{itemize}

\begin{block}{Wann?}

Immer wenn Du merkst, dass Du (alten) Code mehrfach kopierst und an
anderer Stelle einfügst und ihn kaum veränderst, solltest Du darüber
nachdenken, eine Funktion (oder mehrere) daraus zu machen.

\end{block}

\end{frame}

\section{Praxis}\label{praxis}

\begin{frame}[fragile]{Eine Summenfunktion für R}

\begin{block}{Warum wir diese Funktion nicht schreiben sollten}

Eine Summenfunktion gibt es wahrscheinlich in jeder Programmiersprache,
in \textbf{R} heißt sie \texttt{sum()}. Diese Funktion ist besser,
stabiler und schneller, als alles, das wir selbst programmieren können.

\end{block}

\begin{block}{Warum wir es trotzdem tun}

Ich habe die Summenberechnung ausführlich als Beispiel zur
Schleifenprogrammierung in
\href{https://fvafrcu.github.io/programmieren_in_r/\#gute-schleifen}{Programmieren
in R} benutzt.

\end{block}

\end{frame}

\begin{frame}[fragile]{Ein Codestück}

\begin{Shaded}
\begin{Highlighting}[]
\NormalTok{a  <-}\StringTok{ }\KeywordTok{c}\NormalTok{(}\DecValTok{2}\NormalTok{, }\DecValTok{3}\NormalTok{, }\DecValTok{4}\NormalTok{, }\DecValTok{10}\NormalTok{)}
\NormalTok{value <-}\StringTok{ }\DecValTok{0}
\NormalTok{for (a_i in a) \{}
    \NormalTok{value <-}\StringTok{ }\NormalTok{value +}\StringTok{ }\NormalTok{a_i }
\NormalTok{\}}
\KeywordTok{print}\NormalTok{(value)}
\end{Highlighting}
\end{Shaded}

\begin{verbatim}
## [1] 19
\end{verbatim}

\end{frame}

\end{document}

\setbeamertemplate{caption}[numbered]
\setbeamertemplate{caption label separator}{: }
\setbeamercolor{caption name}{fg=normal text.fg}
\beamertemplatenavigationsymbolsempty
\usepackage{lmodern}
\usepackage{amssymb,amsmath}
\usepackage{ifxetex,ifluatex}
\usepackage{fixltx2e} % provides \textsubscript
\ifnum 0\ifxetex 1\fi\ifluatex 1\fi=0 % if pdftex
  \usepackage[T1]{fontenc}
  \usepackage[utf8]{inputenc}
\else % if luatex or xelatex
  \ifxetex
    \usepackage{mathspec}
  \else
    \usepackage{fontspec}
  \fi
  \defaultfontfeatures{Ligatures=TeX,Scale=MatchLowercase}
\fi
% use upquote if available, for straight quotes in verbatim environments
\IfFileExists{upquote.sty}{\usepackage{upquote}}{}
% use microtype if available
\IfFileExists{microtype.sty}{%
\usepackage{microtype}
\UseMicrotypeSet[protrusion]{basicmath} % disable protrusion for tt fonts
}{}
\newif\ifbibliography
\usepackage{color}
\usepackage{fancyvrb}
\newcommand{\VerbBar}{|}
\newcommand{\VERB}{\Verb[commandchars=\\\{\}]}
\DefineVerbatimEnvironment{Highlighting}{Verbatim}{commandchars=\\\{\}}
% Add ',fontsize=\small' for more characters per line
\newenvironment{Shaded}{}{}
\newcommand{\KeywordTok}[1]{\textcolor[rgb]{0.00,0.44,0.13}{\textbf{{#1}}}}
\newcommand{\DataTypeTok}[1]{\textcolor[rgb]{0.56,0.13,0.00}{{#1}}}
\newcommand{\DecValTok}[1]{\textcolor[rgb]{0.25,0.63,0.44}{{#1}}}
\newcommand{\BaseNTok}[1]{\textcolor[rgb]{0.25,0.63,0.44}{{#1}}}
\newcommand{\FloatTok}[1]{\textcolor[rgb]{0.25,0.63,0.44}{{#1}}}
\newcommand{\ConstantTok}[1]{\textcolor[rgb]{0.53,0.00,0.00}{{#1}}}
\newcommand{\CharTok}[1]{\textcolor[rgb]{0.25,0.44,0.63}{{#1}}}
\newcommand{\SpecialCharTok}[1]{\textcolor[rgb]{0.25,0.44,0.63}{{#1}}}
\newcommand{\StringTok}[1]{\textcolor[rgb]{0.25,0.44,0.63}{{#1}}}
\newcommand{\VerbatimStringTok}[1]{\textcolor[rgb]{0.25,0.44,0.63}{{#1}}}
\newcommand{\SpecialStringTok}[1]{\textcolor[rgb]{0.73,0.40,0.53}{{#1}}}
\newcommand{\ImportTok}[1]{{#1}}
\newcommand{\CommentTok}[1]{\textcolor[rgb]{0.38,0.63,0.69}{\textit{{#1}}}}
\newcommand{\DocumentationTok}[1]{\textcolor[rgb]{0.73,0.13,0.13}{\textit{{#1}}}}
\newcommand{\AnnotationTok}[1]{\textcolor[rgb]{0.38,0.63,0.69}{\textbf{\textit{{#1}}}}}
\newcommand{\CommentVarTok}[1]{\textcolor[rgb]{0.38,0.63,0.69}{\textbf{\textit{{#1}}}}}
\newcommand{\OtherTok}[1]{\textcolor[rgb]{0.00,0.44,0.13}{{#1}}}
\newcommand{\FunctionTok}[1]{\textcolor[rgb]{0.02,0.16,0.49}{{#1}}}
\newcommand{\VariableTok}[1]{\textcolor[rgb]{0.10,0.09,0.49}{{#1}}}
\newcommand{\ControlFlowTok}[1]{\textcolor[rgb]{0.00,0.44,0.13}{\textbf{{#1}}}}
\newcommand{\OperatorTok}[1]{\textcolor[rgb]{0.40,0.40,0.40}{{#1}}}
\newcommand{\BuiltInTok}[1]{{#1}}
\newcommand{\ExtensionTok}[1]{{#1}}
\newcommand{\PreprocessorTok}[1]{\textcolor[rgb]{0.74,0.48,0.00}{{#1}}}
\newcommand{\AttributeTok}[1]{\textcolor[rgb]{0.49,0.56,0.16}{{#1}}}
\newcommand{\RegionMarkerTok}[1]{{#1}}
\newcommand{\InformationTok}[1]{\textcolor[rgb]{0.38,0.63,0.69}{\textbf{\textit{{#1}}}}}
\newcommand{\WarningTok}[1]{\textcolor[rgb]{0.38,0.63,0.69}{\textbf{\textit{{#1}}}}}
\newcommand{\AlertTok}[1]{\textcolor[rgb]{1.00,0.00,0.00}{\textbf{{#1}}}}
\newcommand{\ErrorTok}[1]{\textcolor[rgb]{1.00,0.00,0.00}{\textbf{{#1}}}}
\newcommand{\NormalTok}[1]{{#1}}

% Prevent slide breaks in the middle of a paragraph:
\widowpenalties 1 10000
\raggedbottom

\AtBeginPart{
  \let\insertpartnumber\relax
  \let\partname\relax
  \frame{\partpage}
}
\AtBeginSection{
  \ifbibliography
  \else
    \let\insertsectionnumber\relax
    \let\sectionname\relax
    \frame{\sectionpage}
  \fi
}
\AtBeginSubsection{
  \let\insertsubsectionnumber\relax
  \let\subsectionname\relax
  \frame{\subsectionpage}
}

\setlength{\emergencystretch}{3em}  % prevent overfull lines
\providecommand{\tightlist}{%
  \setlength{\itemsep}{0pt}\setlength{\parskip}{0pt}}
\setcounter{secnumdepth}{0}

\title{Funktionen in \textbf{R}}
\subtitle{Ein Einstieg an der Forstlichen Versuchs- und Forschungsanstalt
Baden-Württemberg.}
\author{Dominik Cullmann}
\date{}

\begin{document}
\frame{\titlepage}

\section{Funktionen in R}\label{funktionen-in-r}

\begin{frame}[fragile]{Was sind Funktionen?}

\begin{itemize}
\tightlist
\item
  Funktionen sind Programmkonstrukte, mit denen Du Teile des von Dir
  geschriebenen Codes wiederverwenden kannst (Siehe
  \href{https://de.wikipedia.org/wiki/Funktion_(Programmierung)}{Wikipedia}).
\item
  R-Funktionen kennst Du wahrscheinlich schon: \texttt{sum()},
  \texttt{mean()}, \texttt{summary()}
\end{itemize}

\end{frame}

\begin{frame}{Warum Funktionen?}

\begin{itemize}
\tightlist
\item
  Damit Du Teile des von Dir geschriebenen Codes wiederverwenden kannst.
\item
  Dann musst Du, wenn Du Fehler im Code entdeckst, diesen auch nur an
  einer Stelle korrigieren.
\end{itemize}

\begin{block}{Wann?}

Immer wenn Du merkst, dass Du (alten) Code mehrfach kopierst und an
anderer Stelle einfügst und ihn kaum veränderst, solltest Du darüber
nachdenken, eine Funktion (oder mehrere) daraus zu machen.

\end{block}

\end{frame}

\section{Praxis}\label{praxis}

\begin{frame}[fragile]{Eine Summenfunktion für R}

\begin{block}{Warum wir diese Funktion nicht schreiben sollten}

Eine Summenfunktion gibt es wahrscheinlich in jeder Programmiersprache,
in \textbf{R} heißt sie \texttt{sum()}. Diese Funktion ist besser,
stabiler und schneller, als alles, das wir selbst programmieren können.

\end{block}

\begin{block}{Warum wir es trotzdem tun}

Ich habe die Summenberechnung ausführlich als Beispiel zur
Schleifenprogrammierung in
\href{https://fvafrcu.github.io/programmieren_in_r/\#gute-schleifen}{Programmieren
in R} benutzt.

\end{block}

\end{frame}

\begin{frame}[fragile]{Ein Codestück}

\begin{Shaded}
\begin{Highlighting}[]
\NormalTok{a  <-}\StringTok{ }\KeywordTok{c}\NormalTok{(}\DecValTok{2}\NormalTok{, }\DecValTok{3}\NormalTok{, }\DecValTok{4}\NormalTok{, }\DecValTok{10}\NormalTok{)}
\NormalTok{value <-}\StringTok{ }\DecValTok{0}
\NormalTok{for (a_i in a) \{}
    \NormalTok{value <-}\StringTok{ }\NormalTok{value +}\StringTok{ }\NormalTok{a_i }
\NormalTok{\}}
\KeywordTok{print}\NormalTok{(value)}
\end{Highlighting}
\end{Shaded}

\begin{verbatim}
## [1] 19
\end{verbatim}

\end{frame}

\end{document}
